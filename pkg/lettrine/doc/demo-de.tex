%Format: pdfLaTeX

\documentclass[12pt,a4paper]{article}

\usepackage{lettrine}

\usepackage[T1]{fontenc}
%%% Palatino fonts...
\usepackage{palatino}
%%% ... or EC fonts
%\usepackage{type1ec}

\usepackage{german}

\usepackage{graphicx,color}

\newcommand{\MF}{{\small\sffamily\scshape metafont}}
\newcommand{\MP}{{\small\sffamily\scshape metapost}}

\setlength{\textheight}{680pt}
\setlength{\textwidth}{412pt}
\setlength{\oddsidemargin}{20pt}
\setlength{\topmargin}{-35pt}

\setlength{\parindent}{0pt}
%\addtolength{\parskip}{0.5\baseineskip}
\sloppy

\begin{document}
\thispagestyle{empty}

\begin{center}
\large\bfseries Einige Beispiele zur Anwendung des lettrine-Pakets
\end{center}

\vspace{\baselineskip}
\textit{Einfachste Verwendung (2 Zeilen) :}\\
\verb+\lettrine{E}{in} erstes Beispiel...+

\lettrine{E}{in} erstes Beispiel zeigt das Standardverhalten von lettrine.
Es wird eine Initiale \"uber zwei Zeilen produziert, gefolgt vom Text zwischen
den geschweiften Klammern, der als Kapit\"alchen gesetzt wird. Der weitere
Text umfliesst die Initiale.

\vspace{\baselineskip}
\textit{Lettrine auf einer Zeile (option {\ttfamily\upshape lines=1}) :}\\
\verb+\lettrine[lines=1]{E}{in} zweites Beispiel...+

\lettrine[lines=1]{E}{in} zweites Beispiel zeigt, wie eine Initiale auf
einer einzelnen Zeile aussieht. Auch hier ist der geklammerte Text in
Kapit\"alchen gesetzt.

\vspace{\baselineskip}
\textit{Initiale \"uber drei Zeilen (option {\ttfamily\upshape lines=3}) :}\\
\verb+\lettrine[lines=3]{E}{in} drittes Beispiel...+

\lettrine[lines=3]{E}{in} drittes Beispiel in dem die Initiale \"uber drei
Zeilen gesetzt wird. Beachten Sie die Einr\"uckung der zweiten und dritten
Zeile. Diese kann mit dem Parameter \verb+nindent= + beeinflusst werden. Die
Einr\"uckung der ersten Zeile wird hingegen mit dem Parameter \verb+indent= +
beeinflusst.

\vspace{\baselineskip}
\textit{Initiale vollst\"andig im Randbereich} :\\
\verb+\lettrine[lhang=1, nindent=0pt, lines=3]{V}{erschieben}+

\lettrine[lhang=1, nindent=0pt, lines=3]{V}{erschieben} wir nun im vierten
Beispiel die Initiale in den Randbereich. Dieses Verhalten wird durch den
Parameter \verb+lhang= + gesteuert.

\vspace{\baselineskip}
\textit{Initiale, vergr\"ossert und teilweise im Randbereich} :\\
\verb+\lettrine[lines=3, lhang=0.33, loversize=0.25]{A}{uch}+

\lettrine[lines=3, lhang=0.33, loversize=0.25]{A}{uch}
die Vergr\"osserung der Initiale ist m\"oglich. Die Vergr\"osserung l\"auft
\"uber die Variable \verb+loversize= + . Sie m\"ussen das Ergebnis nicht
unbedingt sch\"on finden. Es sieht nach meiner Meinung besser aus, wie das
vollst\"andige Verschieben in den Randbereich.

\vspace{\baselineskip}
\textit{Setzen wir nun eine %franz\"osisches
         Anf\"uhrungszeichen davor} :\\
\verb+\lettrine[ante=\frqq]{M}{it} dem Parameter ...+

\lettrine[ante=\frqq]{M}{it} dem Parameter \verb+ante= + kann auch ein
beliebiger Text vor die Initiale gesetzt werden. In der Praxis d\"urften
wohl nur Anf\"uhrungszeichen daf\"ur in Frage kommen.

\vspace{\baselineskip}
\verb+\def\lglqq{\raisebox{-\baselineskip}{\glqq}}+\\
\verb+\lettrine[ante=\lglqq]{M}{it} dem Parameter ...+
\def\lglqq{\raisebox{-\baselineskip}{\glqq}}

\lettrine[ante=\lglqq]{M}{it} dem Parameter \verb+ante= + kann auch ein
beliebiger Text vor die Initiale gesetzt werden. In der Praxis d\"urften
wohl nur Anf\"uhrungszeichen daf\"ur in Frage kommen.

\newpage
\vspace{\baselineskip}
\textit{Wir verkleinern die Initiale nun um 10\% und heben sie
um 10\% wegen des {\glqq}Q{\grqq}}
\verb+\lettrine[lines=4, loversize=-.1, lraise=.1]{Q}{ualit"at}+

\lettrine[lines=4, loversize=-.1, lraise=.1]{Q}{ualit"at} hat ihren
Preis.  Und wenn es nur die Zeit ist, um zu lernen wie Sie solche Spielereien
anstellen k\"onnen. Bei den Ergebnissen lohnt sich aber die M\"uhe. Welche
Parameter diesmal was beeinflussen, lasse ich Sie nun selbst herausfinden.
Wie Sie sehen, ragt der Unterstrich des {\glqq}Q{\grqq} nicht in den Text
hinein.

\vspace{\baselineskip}
\textit{Nochmal das {\glqq}Q{\grqq} ohne Anpassungen}
\verb+\lettrine[lines=4]{Q}{ualit"at}+

\lettrine[lines=4]{Q}{ualit\"at} hat ihren Preis.  Und wenn es nur die Zeit
ist, um zu lernen wie Sie solche Spielereien anstellen k\"onnen. Bei den
Ergebnissen lohnt sich aber die M\"uhe. Welche Parameter diesmal was
beeinflussen, lasse ich Sie nun selbst herausfinden. Wie Sie sehen, ragt nun
der Unterstrich des {\glqq}Q{\grqq} in den Text hinein.

\vspace{\baselineskip}
\textit{Verwendung der Option {\ttfamily\upshape slope}}, damit der Text
der Neigung des {\glqq}A{\grqq} folgt:\\
\verb+\lettrine[lines=4, slope=0.6em, findent=-1em,+\\
\verb+          nindent=0.6em]{\A}{uch}...+

\lettrine[lines=4, slope=0.6em, findent=-1em, nindent=0.6em]{A}{uch} eine
Neigung kann angegeben werden. Damit werden die L\"ocher neben geneigten
Buchstaben nicht so gross. Selbst eine negative Neigung ist m\"oglich, damit
bietet auch das {\glqq}V{\grqq} keine Schwierigkeiten mehr. Wie das beim
{\glqq}V{\grqq} aussieht, sehen wir uns beim n\"achsten Beispiel an.

\vspace{\baselineskip}
\textit{Verwendung der Option {\ttfamily\upshape slope}, damit der Text
der Neigung des {\ttfamily\upshape V} folgt; Das {\ttfamily\upshape V} ragt 
zus\"atzlich halb in den Rand hinein 
(Option {\ttfamily\upshape lhang=0.5} :})\\
\verb+\lettrine[lines=4, slope=-0.5em, lhang=0.5, nindent=0pt]+\\
\verb+         {V}{iel} ist...+

\lettrine[lines=4, slope=-0.5em, lhang=0.5, nindent=0pt]{V}{iel} ist hier
nicht anders. Nur die negative Neigung und das Hereinragen in den Rand. Ob
Ihnen das Ergebnis gef\"allt m\"ussen Sie selber entscheiden.  Sie sehen aber,
das das {\glqq}V{\grqq} wirklich keine Schwierigkeiten bietet. Der Unterschied
zum vorhergehenden Beispiel ist nicht besonders gross.

\vspace{\baselineskip}
\textit{\"Andern wir nun die Schriftfamilie f\"ur die Initiale
(hier AvantGarde bold italique):}\\
\verb+\renewcommand{\LettrineFontHook}{\fontfamily{pag}%+\\
\verb+                   \fontseries{bx}\fontshape{it}}+\\
\verb+\lettrine[findent=.3em]{A}{uch} ein Wechsel...+

{% (Aendern des lokalen fonts)
\renewcommand{\LettrineFontHook}{\fontfamily{pag}\fontseries{bx}\fontshape{it}}

\lettrine[findent=.3em]{A}{uch} ein Wechsel der Schriftfamilie ist problemlos
m\"oglich. Hier verwenden wir Avantgarde und setzen mit der Option
\verb+findent= + den horizontalen Abstand des einger\"uckten Texts.
\par}

\vspace{\baselineskip}
\textit{\"Andern wir nun die Schriftfamilie und die Farbe f\"ur die Initiale
(hier yfrak in Grau) :}\\
\verb+\renewcommand{\LettrineFontHook}{\fontfamily{yfrak}\color[gray]{0.5}}+\\
\verb+\lettrine[loversize=0.1]{A}{uch}...+

{% (Aendern des lokalen fonts)
\renewcommand{\LettrineFontHook}{\fontfamily{yfrak}\color[gray]{0.5}}

\lettrine[loversize=0.1]{A}{uch} ein Wechsel der
Schriftfamilie ist problemlos m\"oglich. Hier verwenden wir yfrak,
etwas vergr\"ossert mit der Option \verb+loversize= +, und wir schreiben
die Initiale in Grau mit \verb+\color[gray]{0.5}+.
\par}

\newpage
\begin{center}
\large\bfseries Verwendung eines PostScript-Bildes als Initiale
\end{center}

\vspace{\baselineskip} Wenn die erw\"unschte Initiale nicht als Zeichen eines
Fonts, sondern als Bild im Postscript-Format vorliegt, kann ebenfalls
\verb+\lettrine+ verwendet werden. Es gen\"ugt, 
die Boolsche Variable \texttt{image=true} zu ben\"utzen; z.B. so:

\vspace{.5\baselineskip}
{% Gruppierung, um die  LOKALEN Definitionen zu sch\"utzen
\fontfamily{yfrak}\selectfont\Large
\renewcommand{\LettrineTextFont}{\relax}
\renewcommand{\LettrineFontHook}{\color{red}}
\lettrine[image=true, lines=3, lhang=.2, loversize=.25, %
          lraise=-.05, findent=0.1em, nindent=0em]
{W}{er} reitet so sp"at durch Nacht und Wind?\\
Es ist der Vater mit seinem Kind;\\
Er hat den Knaben wohl in dem Arm,\\
Er fa{\ss}t ihn sicher, er h"alt ihn warm.
\par}

\vspace{\baselineskip} Und hier der zum Beispiel geh\"orende \LaTeX{}--Code:
Das erste Argument von \verb+\lettrine+ war \verb+W+. Die Option \texttt{image=true}
l\"adt dann die Datei \verb+W.eps+. Das Suffix \verb+.eps+ kann -- dank des
Pakets \verb+graphicx.sty+ -- weggelassen werden. \verb+\LettrineFontHook+
enth\"alt den Befehl (\verb+\color{red}+), um in Rot zu schreiben.

\begin{verbatim}
\fontfamily{yfrak}\selectfont\Large
\renewcommand{\LettrineTextFont}{\relax}
\renewcommand{\LettrineFontHook}{\color{red}}
\lettrine[image=true, lines=3, lhang=.2, loversize=.25, %
          lraise=-.05, findent=0.1em, nindent=0em]
{W}{er} reitet so sp"at durch Nacht und Wind?
Es ist der Vater mit seinem Kind;
Er hat den Knaben wohl in dem Arm,
Er fa{\ss}t ihn sicher, er h"alt ihn warm.
\end{verbatim}

Zur Darstellung dieses Beispiels m\"ussen folgende Pakete installiert sein:
\begin{itemize}
\item \verb+graphicx.sty+ und \verb+color.sty+
\item die Schriften \verb+yfrak.pfb+ im type\,1-Format
   von Yannis~\textsc{Haralambous}.
\item Das Paket \verb+blacklettert1+ von Thorsten~\textsc{Bronger}
\end{itemize}

Die gothische Initiale \glqq W\grqq{} in diesem Beispiel k\"onnen Sie mit dem
Programm \MP{} aus den \MF{}-Sourcen und \verb+yinitW.mf+ erzeugen.

Falls Sie eine PDF-Datei erzeugen wollen, m\"ussen Sie die Datei \verb+W.eps+
in eine PDF-Datei \verb+W.pdf+ umwandeln (mit Hilfe von \verb+epstopdf+).

\verb+\lettrine+ unterst\"utzt die Verwendung der Formate:
\texttt{pdf}, \texttt{png}, \texttt{jpeg} oder \MP{} als Initiale.

\vfill
\begin{flushright}
  Deutsche Version Georg \textsc{Wagner}\\
  \texttt{g.wagner@datacomm.ch}\\
  Mai 2003
\end{flushright}

\end{document}

%%% Local Variables:
%%% mode: latex
%%% TeX-master: t
%%% End:
